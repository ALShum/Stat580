\documentclass{article}\usepackage[]{graphicx}\usepackage[]{color}
%% maxwidth is the original width if it is less than linewidth
%% otherwise use linewidth (to make sure the graphics do not exceed the margin)
\makeatletter
\def\maxwidth{ %
  \ifdim\Gin@nat@width>\linewidth
    \linewidth
  \else
    \Gin@nat@width
  \fi
}
\makeatother

\definecolor{fgcolor}{rgb}{0.345, 0.345, 0.345}
\newcommand{\hlnum}[1]{\textcolor[rgb]{0.686,0.059,0.569}{#1}}%
\newcommand{\hlstr}[1]{\textcolor[rgb]{0.192,0.494,0.8}{#1}}%
\newcommand{\hlcom}[1]{\textcolor[rgb]{0.678,0.584,0.686}{\textit{#1}}}%
\newcommand{\hlopt}[1]{\textcolor[rgb]{0,0,0}{#1}}%
\newcommand{\hlstd}[1]{\textcolor[rgb]{0.345,0.345,0.345}{#1}}%
\newcommand{\hlkwa}[1]{\textcolor[rgb]{0.161,0.373,0.58}{\textbf{#1}}}%
\newcommand{\hlkwb}[1]{\textcolor[rgb]{0.69,0.353,0.396}{#1}}%
\newcommand{\hlkwc}[1]{\textcolor[rgb]{0.333,0.667,0.333}{#1}}%
\newcommand{\hlkwd}[1]{\textcolor[rgb]{0.737,0.353,0.396}{\textbf{#1}}}%

\usepackage{framed}
\makeatletter
\newenvironment{kframe}{%
 \def\at@end@of@kframe{}%
 \ifinner\ifhmode%
  \def\at@end@of@kframe{\end{minipage}}%
  \begin{minipage}{\columnwidth}%
 \fi\fi%
 \def\FrameCommand##1{\hskip\@totalleftmargin \hskip-\fboxsep
 \colorbox{shadecolor}{##1}\hskip-\fboxsep
     % There is no \\@totalrightmargin, so:
     \hskip-\linewidth \hskip-\@totalleftmargin \hskip\columnwidth}%
 \MakeFramed {\advance\hsize-\width
   \@totalleftmargin\z@ \linewidth\hsize
   \@setminipage}}%
 {\par\unskip\endMakeFramed%
 \at@end@of@kframe}
\makeatother

\definecolor{shadecolor}{rgb}{.97, .97, .97}
\definecolor{messagecolor}{rgb}{0, 0, 0}
\definecolor{warningcolor}{rgb}{1, 0, 1}
\definecolor{errorcolor}{rgb}{1, 0, 0}
\newenvironment{knitrout}{}{} % an empty environment to be redefined in TeX

\usepackage{alltt}
\usepackage{url}
\usepackage{amsfonts}
\usepackage{amsmath}
\usepackage[margin=0.70in]{geometry}
\setlength{\parindent}{0in}

\title{Stat580 - Homework 2}
\author{Alex Shum}
\IfFileExists{upquote.sty}{\usepackage{upquote}}{}
\begin{document}
\maketitle

\section{Problem 1}
\begin{align*}
f(x,y) = f(y|x)f(x) = \frac{f(y|x)}{\frac{1}{f(x)}} = \frac{f(y|x)}{\int \frac{f(y)}{f(x)} dy} = \frac{f(y|x)}{\int \frac{\frac{1}{f(x)}}{\frac{1}{f(y)}}dy} = \frac{f(y|x)}{\int \frac{\frac{f(x,y)}{f(x)}}{\frac{f(x,y)}{f(y)}}dy} = \frac{f(y|x)}{\int \frac{f(y|x)}{f(x|y)}dy}
\end{align*}
Thus we can see that the joint $f(x,y)$ can be written in terms of the two conditions: $f(x|y)$ and $f(y|x)$.

\section{Problem 2}
We have that $Y \sim \chi^2_p$ and we have that $\mathbf{X} \sim MVN(0, \mathbf{I})$.  This means that $X_1, X_2 \dots X_n \sim N(0,1)$ and we have the following:

\begin{align*}
f_y(y) &= \frac{1}{2^{k/2}\Gamma(k/2)}y^{k/2 - 1}e^{-y/2} \\
f_\mathbf{x}(\mathbf{x}) &= (2\pi)^{-k/2} e^{-1/2 ||\mathbf{x}||^2}
\end{align*}

We need to perform the following transform: $\mathbf{Z} = \frac{\sqrt{Y} \mathbf{X}}{||\mathbf{X}||}$.  This means that we have $Z_i = \frac{\sqrt{Y} X_i}{||\mathbf{X}||}$.  We also define $W = \frac{\sqrt{Y}}{||X||}$.  From the previous definition we have that $Z_i = WX_i$.  We have the following inverse transformations: $X_i = \frac{Z_i}{W}$ and we have that $Y = W^2||\mathbf{X}||^2 = W^2 \Sigma_{i=1}^n X_i^2 = W^2 \Sigma_{i=1}^n \frac{W^2X_i^2}{W^2} = \Sigma_{i=1}^n W^2X_i^2 = \Sigma_{i=1}^nZ_i^2$.  The jacobian is as follows:

\begin{align*}
J = \begin{bmatrix}
\frac{\partial X_1}{\partial Z_1} & \frac{\partial X_1}{\partial Z_2} & \dots & \frac{\partial X_1}{\partial Z_n} & \frac{\partial X_1}{\partial W} \\
\frac{\partial X_2}{\partial Z_1} & \frac{\partial X_2}{\partial Z_2} & \dots & \frac{\partial X_2}{\partial Z_n} & \frac{\partial X_2}{\partial W} \\
\vdots & \vdots & \ddots & \vdots & \vdots \\
\frac{\partial X_n}{\partial Z_1} & \frac{\partial X_n}{\partial Z_2} & \dots & \frac{\partial X_n}{\partial Z_n} & \frac{\partial X_n}{\partial W} \\
\frac{\partial Y}{\partial Z_1} & \frac{\partial Y}{\partial Z_2} & \dots & \frac{\partial Y}{\partial Z_n} & \frac{\partial Y}{\partial W}
\end{bmatrix} = 
\begin{bmatrix}
\frac{1}{W} & 0 & \dots & 0 & \frac{-Z_1}{W^2}\\
0 & \frac{1}{W} & \dots & 0 & \frac{-Z_2}{W^2}\\
\vdots & \vdots & \ddots & \vdots & \vdots \\
0 & 0 & \dots & \frac{1}{W} & \frac{-Z_n}{W^2}\\
2Z_1 & 2Z_2 & \dots & 2Z_n & 0
\end{bmatrix}
\end{align*}

If we take the determinant of this jacobian we have that $|J| = \frac{2}{W^{n + 1}}||Z||^2$.  We assume they are independent thus: $f_{W,Z}(w,z) = f_X(Z/W) f_Y(||Z||) |J|$.  If we integrate out $W$ we have that $Z$ is a MVN distribution (this took many pages of scratch work).

\section{Problem 3}
See readme.txt file and code in /class/stat580/ashum/homework2/function\_passing

\section{Problem 4}
See readme.txt file and code in /class/stat580/ashum/homework2/CCS\_run

\section{Problem 5}
Instead of using $k+1$ for loops for all the summations my original idea was to use an array of bits or boolean variables.  Each element in this array represents one of the $X_i$ variables and is either 0 or 1.  We would have a for loop to iterate from $i=0$ to $i=2^{k + 1} - 1$, where $k + 1$ is the same number of $X_i$ variables.  For each iteration we would convert $i$ to a bit array (an array of bits corresponding to the binary representation of $i$).  We could then iterate through this bit array for the values of $X_i$ as we take the product in the likelihood.  When we complete all $2^{k + 1} - 1$ iterations this will try all combinations $0$ and $1$ for the $X_i$s.  However, the C language makes it quite difficult to create a bit array from an integer.\\
\\
Instead we could convert the integer $i$ in our for loop into binary representation.  We could then check if the $j$th digit in the binary representation is 1 or 0 by using a second binary number that is all $0$ with a single $1$ in the $j$th position and using the $AND$ operator on the two binary integers.  If the resulting binary integer is zero then in our binary representation of integer $i$ at position $j$ is 0 and otherwise it will be non-zero.  The shift operator comes into play when we want to check all binary digits of integer $i$.  We would have a second binary number with all $0$ and a $1$ in the first position.  We could iterate through by shifting the only non-zero bit and apply the $AND$ operator with the binary representation of $i$; this would be equivalent to iterating through each position of a bit array as described above.

\section{Problem 6}
See readme.txt file and code in /class/stat580/ashum/homework2/power\_iteration

\end{document}
